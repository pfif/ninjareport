\chapter{Conclusion}
Ce stage a eu une experience positive sur mes compétences informatiques. En plus des classiques nouveaux langages, nouveaux frameworks, j'ai aujourd'hui une vision du monde informatique qui a complemement changé. La façon dont j'appréhende un projet, le temps de travail, la architecture d'un système, le monde professionel, tout ceci a été remis en cause. Je suis maintenant, je le crois, bien plus efficace. Et j'ai de nombreuses pistes pour continuer mon amélioration.

Listons ces points techniques où j'ai fais des progrès : JavaFX, Java 8, Maven, Controls FX, GitHub et le plus important : Git. 

Le projet a bien avancé, John et moi avons fait un travail correct et construit un projet durable, avec une communauté d'utilisateurs satisfaits. La route est encore longue pour \textit{Topsoil}, de nombreuses fonctionnalités petites ou grandes restent à être codées, de nouveaux utilisateurs doivent être gagnés, de nouveaux développeurs doivent contribuer. Déja des dates et des conférences sont prévues pour cet outil que je suivrai après mon départ.

Les seules difficultés que j'ai eu pendant ce stage m'ont rendu plus fort et qualifié et mon seul regret est que cela se termine déjà.\\

Ce stage fut un très agréable moment de ma vie. J'ai rencontré à Charleston des personnes fantastiques, ai découvert une autre culture, une autre façon d'éduquer, de travailler, ai vu des paysages magnifiques. Mon anglais de conversation s'est considerablement amélioré mon anglais écrit a lui aussi bénéficié de ces trois mois. Ma vision du futur a completment été en perspective par cette experience.

Une chose est certaine : Charleston, I'll be back!
