\chapter{Knowledge to be gained from \textit{The Visual Display of Quantitative Information} facing \textit{CIRDLES' Tufte}}

\section{Introduction}
Topsoil is the latest projet from the CIRDLES' Lab. It allow geochronologists to plot various graphs needed for the analysis of aliquots. This software is based on the library JavaFX that gives Java the ability to display user interface.
The first and current version of Topsoil is using the ploting abilities of JavaFX. But as we copied and rewrite part of it to make it fit our needs, we quickly realize that it had too much flaws to be a long term solution. And being a long term solution is one of this project objective.
After long and laborious search, John did not found a good library that could plot graphics for us. The solutions were either too specific, not well-designed enough or not compatible with JavaFX. From that moment on, it become clear that we needed to create our own graphic library. 

Designing, coding and maintaining this kind of library is a lot of work, worth the result. But it is preferable that a team document itself before starting a project of this scale. 
In that regard, a book can bring a lot of information : \textit{The Visual Display of Quantitive Information} by Edward Tufte. This document describe several information to this book and how they might be useful to the project. %GOD YOU HAVE TO MODIFY THAT

This document will be structured in three parts : 
\begin{enum}
\item First of all, I will give a very brief summary of the book, allowing the reader to feel the tone of the book.
\item Then I will give a number of chart that the library should be able to produce.
\item Finally, I will give a number of Tufte's rules for the diagram and some ideas on how the library should enforce or encourage to enforce them. 
\end{enum}
%Please note very wide not to follow right away
\section{A quick summary of this book}
This book is a manifesto. It assert that chart are powerful tools of communication, explain why and how to make great graphics.
Along the book, Tufte find regrettable that plot are underused in many kinds of publication, like textbooks, papers or journals. He also list a few reason of this underusage.
%What is part I/What is part II

\section{A list of graphic}
It has been decided that our library should support a lot of type of graphics. To expend the range of known type by the team and gather documentation in one place, a number of chart along with the features needed in order to support them will be given here.

\subsection{Data Map chart}
Laying out data on a map can with no doubt, can carry an enormous volume of data and have a lot of reading levels.

%Complexe line representing the sea.
To implement this map, the \textit{Tufte} library's graph background would have to be be filled up with a map\footnote{It could be useful to look into interoperable map format, or even map libraries}, integrating name of landmarks. Symbols of all kind could be added, from a single point on a city to complex lines representing the sea like here :

%Cancer map
An other important feature to consider for the library to implement this map is the binding of data to zone on the map in the background.

%French export
To draw this map, several layers of the map have to be known by the rendered line. The informations passed to draw theses are :
\begin{enum}
\item Position of the begining
\item Position of the end
\item A quantity
\end{enum}
The quantity is then turned into the width of line, which will be drawn from its begining to its end, without crossing earth.
The begining and the end must also fit closely the earth. These are the two reasons why the rendered elements should be awared of the map and its shape.

\subsection{Time-Space chart}
%NY City Weather
Several informations can be gained from that plot.

First, notice the indications pointing on the main curve (like "LINE INDICATE THE NORMAL LOW"). Tufte love their usefulness. We should find a way to make them point to either a point of a serie of point, or to nothing. Also, allow the automation to print the data they are pointed to
Secondly, notice that the temperature axis is reproduced on the two sides. Printing the same axis mutliple time on the same plot should be a possibility. 
Thirdly, there is one subgraph drawn for each month representing the difference between the actual precipitation and the normal.
Last, but not least : the several pieces of data are linked to the same x-axis and to different y-axis.

%Playfair : "Sharing at one wiew the Price of Quarter of Wheat ...
Concernine this diagram, I cannot think of more than one axis of reflection for now. There are two sets of data that have there own space, making the plot more readable. Maybe we could arrange plot of data to be aware of their sibblings, or make the PlotArea smarter when it is outlining two set of data.

%http://worldtracker.org/media/library/Science/Science%20Magazine/science%20magazine%201981-1982/Science%201981-1982/root/data/Science_1981-1982/pdf/1982_v216_n4550/p4550_1086.pdf

\subsection{Time-Space chart}
%Teh Napoleon Chart
The very interesting construction of the diagram is here made in two times. The following data is given to draw the path of Napoleon's army in russia, for each step : the two numbers of the location of the step, the number of soldier (transformed in the width) that was at that moment in the army, and the date which give the order\footnote{There is also the travel direction of the army that give the color, but it is not relevant in this context}.

There are two things to say about that. The plot of a serie of data can not be made of several plot of the different point that it is made of : there is also information about the link between all those plot. For exemple, here, if we ploted all the plot one at the time, the location would be drawn, but not the path between two of those location.
Let's now take a look at the bottom of the chart, where the temperature of each steps is represented. The x-axis is simply representing the temperatures. The y-axis on the other hand is "the step", and is given by the first and main plot of the diagram. This is a very interesting use case.
