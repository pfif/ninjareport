\chapter{Introduction}

Durant mon stage, j'ai visionné plusieurs vidéos produitent par un géochronologiste nommé Simon Bowring%mettre un lien vers la vidéo
. Lui et son équipe ont une façon bien à eux de piquer votre intérêt dès le début de la vidéo : Rappeler au public que l'être humain n'est qu'un animal comme les autres, que nous sommes sans cesse menacé d'extinction par un grand nombres de catastrophes et qu'il est necessaire que nous nous rappelions que nous avons été sur cette planète une fraction tellement infime de son existance que nous ne sommes pas si importants que nous aimerions le croire. Si l'on représente l'integralité de l'histoire de la terre sur un calendrier, nous apparaissons Le 365ième jour vers 22 heures !

Ceci fait, il continue en nous présentant ses recherches. Je trouve que c'est une excelente façon d'introduire le sujet.\\

Je suis actuellement en stage au laboratoire \textit{CIRDLES}, partie du \textit{College de Charleston}, situé dans la ville de Charleston aux États-Unis. %et j'ai la plus merveilleuses des copines!
Du 14 avril au 2 Juillet 2014, j'aide au développement d'un logiciel nommé \textit{Topsoil}, qui permet à des géochronologistes de tracer des graphiques dont ils ont besoins. La géochronologie est une sous branche de la géologie qui cherche à dater différents objets terrestres. 

J'ai travaillé sous la direction du Docteur Bowring et en collaboration avec John Zeringue. Nous deux principales missions étaient de coder une version primitive du système, puis la montrer à des utilisateurs potentiel et orienter le projet selon leurs remarques.

C'était un stage très agréable, plein de découverte, très instructif et durant lequel j'ai rencontré un grand nombre de personnes formbidables. Sans plus attendre, partons à la découverte du developpement de \textit{Topsoil}.
